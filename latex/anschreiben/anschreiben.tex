\documentclass[accentcolor=tud7a]{tudletter}
\usepackage{etex}
\usepackage[ngerman]{babel}
\usepackage[utf8]{inputenc}
\usepackage{url}
\usepackage{wasysym}
\usepackage{enumitem}
\usepackage{ifthen}
\input ean13
\usepackage{courier}
\setkomavar{frominstitution}{Fachschaft Informatik}
\setkomavar{fromlogo}{\includegraphics{das-wesen-der-informatik}}
\setkomavar{fromname}{Feedbackteam}
\setkomavar{fromaddress}{Hochschulstr. 10\\64289 Darmstadt}
\setkomavar{fromphone}{06151 16-25522}
\setkomavar{fromemail}{feedback@d120.de}

\setkomavar{signature}{i.A. Fachschaft Informatik}

\setkomavar{shortfromname}{Feedbackteam}

% definierte Parameter:
% Name des Empfängers #1
% Adresse des Empfängers #2
% Veranstaltungsname #3
% Anzahl Fragebögen #4
% Sprache Fragebögen #5
% Typ #6
% Kennung #7
% Freie Frage1 #8
% Freie Frage2 #9

\begin{document}
\renewcommand*{\adrentry}[9]{%in eckigen Klammern Anzahl der Parameter
\setkomavar{myref}{#7}
\setkomavar{subject}[]{Evaluation \textbf{#6: #3}}
\begin{letter}{Hauspost\\#1\\#2\\#3\\
\barheight=1cm
%\X=0.396mm
\EAN #7
}

\opening{Sehr geehrte Damen und Herren,}

als Anlage zu diesem Schreiben finden Sie die Fragebögen zur Evaluation der Lehre am Fachbereich 
Informatik. In diesem Schreiben haben wir alle wichtigen Informationen für Sie zusammengefasst.\\

\noindent\textbf{Disclaimer:}
\begin{itemize}[partopsep=0pt, topsep=-\parskip, parsep=0pt, itemsep=0pt, leftmargin=*, rightmargin=130px]
\item{}Evaluation bitte in der Woche \input{erhebungswoche.inc}durchführen
\item{}Evaluationsrichtlinien der TU Darmstadt untersagen eigene Auswertung
\item{}Wenn die Bögen nicht per Hauspost zurückgebracht werden, die leeren Bögen außerhalb des Umschlages transportieren.
\item{}Wenn die Bögen erst nach der Evaluationswoche bei uns eingehen, kann die Bearbeitung länger dauern.
\end{itemize}
%\vspace{5mm}
\textbf{Checklist:}
\begin{itemize}[partopsep=0pt, topsep=-\parskip, parsep=0pt, itemsep=0pt, leftmargin=*, rightmargin=130px]
\ifthenelse{\equal{#6}{Vorlesung und Übung}}{
\item[\Square]\textit{Vorlesung mit individueller Tutorenevaluation}: Übungsbögen und Anschreiben an Tutoren verteilen, in Übungsstunden evaluieren lassen, Rückgabe pro Gruppe}{}
\item[\Square]Studierenden \underline{vorab} genauen Evaluationstermin Ihrer Veranstaltung mitteilen
\item[\Square]möglichst in der ersten Veranstaltung in der Erhebungswoche evaluieren
\item[\Square]Rückseite des Anschreibens enthält unsere Anschrift für die Rücksendung, bzw. beiliegendes Etikett über das vorhandene kleben
\item[\Square]Studierenden bei der Evaluation die freien Fragen (siehe Rückseite) mitteilen (per Tafel, Beamer etc.)
\item[\Square]bei kleinen Veranstaltungen die Studierenden auffordern, zur Wahrung der Anonymität, die Angabe zum Geschlecht nicht auszufüllen
\item[\Square]\underline{vor dem Austeilen} der Fragebögen zwei Studierende finden, welche alle Bögen zur Fachschaft Informatik (Gebäude S2|02 Raum D120) zurückbringen (sollte niemand im Fachschaftsraum sein, Bögen in den Briefkasten des Studiensekretariats vor C117 werfen) \textbf{oder}
ausgefüllte Fragebögen \underline{direkt} per Hauspost im Umschlag an uns zurücksenden
\item[\Square]leere Fragebögen bitte ebenfalls an uns zurücksenden
\end{itemize}
\vspace{5mm}
\textbf{Bitte Wenden}

\encl{Typ: #6, Fragebögen: \ifthenelse{\equal{#6}{Vorlesung mit Übung}}{je }{}#4, Sprache: #5}
\end{letter}

\setkomavar{myref}{#7}
\setkomavar{subject}[Betreff: ]{Antwort: \textbf{#3}}

\begin{letter}{Hauspost -- \textbf {Antwort}\\
Fachschaft Informatik\\
z.Hd. Feedback-Team\\
S2 02 -- D120\\
#3\\
\barheight=1cm
\EAN #7
}

\opening{}

\noindent Freie Frage 1: #8\\
\noindent Freie Frage 2: #9\\
\\
\noindent Diese Seite bitte für das Zurücksenden der ausgefüllten und \underline{leeren} Bögen oben auf den Stapel legen. Wir benötigen den Barcode!\\
\\
\noindent Weitere Informationen finden Sie auf unserer Webseite\\ \url{http://d120.de/feedback-new/} \\
Mit Fragen oder Anregungen wenden Sie sich bitte per E-Mail an\\
\url{feedback@d120.de}\\
\closing{Vielen Dank für Ihre Mühe,}

\encl{Typ: #6, Fragebögen: \ifthenelse{\equal{#6}{Vorlesung mit Übung}}{je }{}#4, Sprache: #5}
\end{letter}
}

\input{veranstalter.adr}
\end{document}
